\documentclass{article}
\usepackage{graphicx, amssymb}
\usepackage{amsmath}
\usepackage{amsfonts}
\usepackage{amsthm}
\usepackage{kotex}
\usepackage{bm}
\usepackage{hyperref}
\usepackage{xcolor}
\usepackage{mathrsfs}
\usepackage{mathtools}
\usepackage{physics}

\textwidth 6.5 truein 
\oddsidemargin 0 truein 
\evensidemargin -0.50 truein 
\topmargin -.5 truein 
\textheight 8.5in

\DeclareMathOperator{\cc}{\mathbb{C}}
\DeclareMathOperator{\rr}{\mathbb{R}}
\DeclareMathOperator{\bA}{\mathbb{A}}
\DeclareMathOperator{\fra}{\mathfrak{a}}
\DeclareMathOperator{\frb}{\mathfrak{b}}
\DeclareMathOperator{\frm}{\mathfrak{m}}
\DeclareMathOperator{\frp}{\mathfrak{p}}
\DeclareMathOperator{\slin}{\mathfrak{sl}}
\DeclareMathOperator{\Lie}{\mathsf{Lie}}
\DeclareMathOperator{\Alg}{\mathsf{Alg}}
\DeclareMathOperator{\Spec}{\mathrm{Spec}}
\DeclareMathOperator{\End}{\mathrm{End}}
\DeclareMathOperator{\rad}{\mathrm{rad}}
\newcommand*\Laplace{\mathop{}\!\mathbin\bigtriangleup}
\newcommand{\id}{\mathrm{id}}
\newcommand{\Hom}{\mathrm{Hom}}
\newcommand{\Sch}{\mathbf{Sch}}
\newcommand{\Ring}{\mathbf{Ring}}
\newcommand{\T}{\mathcal{T}}
\newcommand{\B}{\mathcal{B}}
\newcommand{\Mod}[1]{\ (\mathrm{mod}\ #1)}
\newtheorem{lemma}{Lemma}
\newtheorem{theorem}{Theorem}
\newtheorem{proposition}{Proposition}

\begin{document}


\title{Partial Differential Equation - HW 2}
\author{SungBin Park, 20150462} 

 \maketitle

\section*{Problem 1}
\begin{enumerate}
	\item [Step 1.] Let a linear transformation from $\rr^{n+1}$ to $\rr^{n+1}$ sending $s\mapsto r^2 s$ and $y_i\mapsto ry_i$, then $T$ sends $E(0, \bm{0}, r)$ to $E(0, \bm{0}, 1)$. Then
	\begin{equation*}
	\begin{split}
	\frac{1}{4}\iint_{E(0, \bm{0}, 1)}\frac{\abs{y}^2}{s^2}dyds&=\frac{1}{4}\iint_{E(0, \bm{0}, r)}\frac{\abs{y}^2/r^2}{s^2/r^4}\frac{1}{r^{n+2}}dyds \\
	&=\frac{1}{4r^n}\iint_{E(0, \bm{0}, r)}\frac{\abs{y}^2}{s^2}dyds
	\end{split}
	\end{equation*}
    \item [Step 2.] Since $e^x\leq 1$ if $x\leq 0$,
    \begin{equation*}
        \begin{split}
            &\frac{1}{\left(4\pi(-s)\right)^{n/2}}e^{\frac{\abs{y}^2}{4s}}\geq 1 \text{ for }s\leq 0\\
            &\Leftrightarrow 4\pi(-s)\leq 1 \text{ and }e^{\frac{\abs{y}^2}{4s}}\geq \left(4\pi(-s)\right)^{n/2} \text{ and } s\leq 0\\
            &\Leftrightarrow -\frac{1}{4\pi}\leq s\leq 0 \text{ and } \abs{y}^2\leq 2sn\ln{(-4\pi s)}
        \end{split}
    \end{equation*}
    Therefore, $E(0,0,1)=\left\{(s,y)\in R^{n+1}:-\frac{1}{4\pi}\leq s\leq 0 , \abs{y}^2\leq -2sn\ln{\left(\frac{1}{-4\pi s}\right)}\right\}$
    
    \item[Step 3.] Since $\abs{y}^2\leq -2sn\ln\left(\frac{1}{-4\pi s}\right)$ is the ball with radius $\left(-2sn\ln\left(\frac{1}{-4\pi s}\right)\right)^{1/2}$, 
    \begin{equation*}
    \begin{split}
        \iint_{E(0, \bm{0}, 1)} \frac{\abs{y}^2}{s^2}dyds&=\int_{-1/4\pi}^0\int_{B_s}\frac{\abs{y}^2}{s^2}dyds \text{ (By Fubini theorem)}\\
        &=\int_{-1/4\pi}^0 \frac{1}{s^2} \int_{S^{n-1}}\int_0^{\left(-2sn\ln\left(\frac{1}{-4\pi s}\right)\right)^{1/2}} r^{n+1}drd\sigma ds \text{ (By Spherical Coord.)}\\
        &=\int_{-1/4\pi}^0 \frac{1}{s^2}\frac{\frac{n}{2}\pi^{n/2}}{\Gamma(\frac{n}{2}+2)} \left(-2sn\ln\left(\frac{1}{-4\pi s}\right)\right)^{n/2+1} ds
    \end{split}
    \end{equation*}
    Let $t=\ln\left(\frac{1}{-4\pi s}\right)$, then 
    \begin{equation*}
        \begin{split}
            &\frac{1}{4}\left(\frac{n}{2\pi}\right)^{n/2+2}\frac{\pi^{n/2}}{\Gamma(n/2+2)}\int_{0}^\infty \frac{(4\pi)^2}{e^{-2t}} \left(e^{-t}t\right)^{n/2+1} e^{-t}dt \\
            &=4\left(\frac{n}{2}\right)^{n/2+2}\frac{1}{\Gamma(n/2+2)}\int_0^\infty t^{n/2+1}e^{-(n/2)t}dt \\
            &=4\left(\frac{n}{2}\right)^{n/2+2}\frac{1}{\Gamma(n/2+2)}\left(\frac{2}{n}\right)^{n/2+2}\Gamma(n/2+2)=4.
        \end{split}
    \end{equation*}
    Therefore, $\frac{1}{4}\iint_{E(0, \bm{0}, 1)} \frac{\abs{y}^2}{s^2}dyds=1$.
\end{enumerate}
Summarising the results, $\frac{1}{4r^n}\iint_{E(0, \bm{0}, r)}\frac{\abs{y}^2}{s^2}dydx=1$ for all $r>0$.

\section*{Problem 2}
By change of variable shifting $x$ and $t$ to 0, we can rearrange the formula by \begin{equation*}
\frac{1}{r^n}\iint_{E(t,x;r)}u(s,y)\frac{\abs{s-y}^2}{\abs{t-s}^2}dyds=\iint_{E(0,\bm{0};r)}u(r^2s,ry)\frac{\abs{y}^2}{s^2}dyds
\end{equation*}
Let $\phi(r)$ be the RHS of above equation. Our strategy is showing that $\phi(r)$ is constant function, and it means $\phi(r)=\lim\limits_{r\rightarrow 0} \phi(r)=r^n u(x,t)$ using continuity of $u$ and the result from problem 1. I'll write $E(r)=E(0,\bm{0};r)$.
Computing $\phi'(r)$,
\begin{equation*}
\begin{split}
\phi'(r)&=\dv{r}\iint_{E(1)} u(r^2s,ry)\frac{\abs{y}^2}{s^2}dyds \\
&= \iint_{E(1)}(2rsu_s(r^2s, ry)+y\cdot \nabla_y u)\frac{\abs{y}^2}{s^2}dyds \text { (Since }\frac{\abs{y}^2}{s^2}\in L^1(E(1)) \text{ and } u\in C(\overline{E(1)}) \\
&= \frac{1}{r^{n+1}}\iint_{E(r)}(2su_s(s,y)+y\cdot \nabla_y u)\frac{\abs{y}^2}{s^2}dyds
\end{split}
\end{equation*}
Let the first term $A$ and last term $B$.

For simple calculation, I'll introduce a function
\begin{equation*}
\varphi(s, y)= -\frac{n}{2}\ln(-4\pi s)+\frac{\abs{y}^2}{4s}+n\ln r
\end{equation*}
Then, we can get a relation $\abs{y}^2=\sum\limits_{i=1}^n 2sy_i\varphi_{y_i}$, and
\begin{equation*}
\begin{split}
r^{n+1}A&=\iint_{E(r)} 4u_s \sum\limits_{i=1}^n y_i \varphi_{y_i} dyds \\
&=\iint_{E(r)} 4 \sum\limits_{i=1}^n \left(\pdv{y_i}\left(y_iu_s\varphi\right)-u_s\varphi-y_i\varphi_{sy_i}\right) dyds \\
&=-4\iint_{E(r)} nu_s\varphi + \varphi\sum\limits_{i=1}^n u_{sy_i} y_i  dyds
\end{split}
\end{equation*}
Since
\begin{equation*}
\begin{split}
\iint_{E(r)} \sum\limits_{i=1}^n \pdv{y_i}\left(y_i u_s\varphi\right) dyds&=\sum\limits_{i=1}^n \int_{-\frac{r^2}{4\pi}}^0 \int_{B\left(\bm{0}, -2ns \ln{\left(-\frac{4\pi s}{r^2}\right)}\right)}  \pdv{y_i}\left(y_i u_s\varphi\right) dyds \\
&=\sum\limits_{i=1}^n \int_{-\frac{r^2}{4\pi}}^0 \int_{\partial B\left(\bm{0}, -2ns \ln{\left(-\frac{4\pi s}{r^2}\right)}\right)}  \left(y_i u_s\varphi\right)\nu^i dyds \\
&=0
\end{split}
\end{equation*}
for $\varphi\equiv 0$ in $\partial E-\{0,\bm{0}\}$.($\nu^i$ is ....)

Since $\pdv{s}\left(u_{y_i}y_i\varphi\right)=u_{sy_i}y_i\varphi+u_{y_i}\varphi$ ...
\begin{equation*}
\iint_{E(r)}\pdv{s}\left(u_{y_i}y_i\varphi\right)dyds=\int_{\abs{y}^2\leq\frac{nr^2}{2\pi e}}\int_{s_0(y)}^{s_1(y)}\pdv{s}\left(u_{y_i}y_i\varphi\right)dsdy=0
\end{equation*}
where $s_0(y)$ and $s_1(y)$ is the $s$ ... for $y\neq \bm{0}$ because $s_0(y)\neq 0$.

We know that
\begin{equation*}
\begin{split}
-4\iint_{E(r)} nu_s\varphi - \varphi\sum\limits_{i=1}^n u_{sy_i} y_i  dyds &= -4\iint_{E(r)} nu_s\varphi - \varphi\sum\limits_{i=1}^n u_{y_i} y_i \varphi_s dyds \\
&=-4\iint_{E(r)} nu_s\varphi - \sum\limits_{i=1}^n u_{y_i} y_i \left(-\frac{n}{2s}-\frac{\abs{y}^2}{4s^2}\right) dyds \\
&=\iint_{E(r)} -4nu_s\varphi - \frac{2n}{s}\sum\limits_{i=1}^n u_{y_i} y_i dyds - B.
\end{split}
\end{equation*}

\section*{Problem 3}
\begin{enumerate}
\item[Step 1.] To satisfy $u_t-u_{xx}=0$ for $t>0, x\in \rr$, (...Assume convergence) 
\begin{equation*}
u_t-u_{xx}=\sum_{j=0}^\infty \left(\pdv{g_j(t)}{t}-(j+1)(j+2)g_{j+2}(t)\right)x^{j}=0.
\end{equation*}
By uniqueness of power series(...), $\pdv{g_j(t)}{t}-(j+1)(j+2)g_{j+2}=0$ for all $j\geq 0$ make u satisfies $u_t-u_{xx}=0$.

To satisfy initial condition $u(0, x)=0$ for $x\in \rr$, $\sum_{j=0}^\infty g_j(0)x^j=0$, implying $g_j(0)=0$ for all $j$.

\item[Step 2.] I'll use induction for proof. Let $g_j$ be
\begin{equation}\label{Eq:3.1}
g_j(t)=\begin{cases}
0 & j\text{ is odd.} \\
\frac{1}{(2(j/2))!}\frac{d^{(j/2)}}{dt^{(j/2)}}g(t) & j\text{ is even.}
\end{cases}
\end{equation}
for $j<2n$. The starting case is given in the problem. For $j=2n$,
\begin{equation*}
g_{2n}=\frac{1}{(2n-1)(2n)}\pdv{g_{2n-2}(t)}{t}=\frac{1}{2n!}\frac{d^{n}}{dt^{n}}g(t).
\end{equation*}
For $j=2n+1$,
\begin{equation*}
g_{2n+1}=\frac{1}{(2n)(2n+1)}\pdv{g_{2n-1}(t)}{t}=0.
\end{equation*}
Therefore, \eqref{Eq:3.1} is valid.
\item[Step 3.] Since $\frac{1}{z^2}$ is holomorphic without $z= 0$, $e^{-\frac{1}{z^2}}$ is holomorphic without $z=0$. For $z\neq 0$, there is a open disk of radius $r>0$ such that the disk does not contain $0$. Then by Cauchy integral formula,
\begin{equation*}
\abs{\frac{d^k}{dt^k}g(t)}\leq \frac{k!}{2\pi}\int_0^{2\pi} \abs{\frac{g(t+Re^{i\theta})}{R^{n}}}d\theta
\end{equation*}
sibal....

\item[Step 4.] Let $g_i$ as in Step 2. ..
\end{enumerate}

\section*{Problem 4}
Consider the following PDE:
\begin{equation}\label{Eq:P4-1}
u_t-\Laplace_x u=f(t,x)e^{ct}~~\text{for }t>0,x\in\rr^n,~~u(0,x)=g(x)
\end{equation}
Then, $f(t,x)e^{ct}$ and $g(x)$ are smooth and have compact supports. Therefore, there exists a unique solution $u$ satisfying \eqref{Eq:P4-1}:
\begin{equation*}
u(x,y)=\int_{\rr^n}\Phi(x-y,t)g(y)dy+\int_0^t\int_{\rr^n}\Phi(x-y,t-s)f(y,s)e^{ct}dyds
\end{equation*}
Let's consider $u'=ue^{-ct}$, then
\begin{equation*}
\begin{split}
&u'_t-\Laplace_x u'+cu'=u_te^{-ct}-cue^{ct}-\Laplace_x ue^{ct}+cue^{ct}=(u_t-\Laplace_x u)e^{ct}=f(t,x) \\
&u'(0,x)=ue^{c\cdot 0}=u=g(x)
\end{split}
\end{equation*}
for $t>0$ and $x\in \rr^n$. By the uniqueness of the solution ....
\begin{equation*}
e^{ct}\left(\int_{\rr^n}\Phi(x-y,t)g(y)dy+\int_0^t\int_{\rr^n}\Phi(x-y,t-s)f(y,s)e^{ct}dyds\right)
\end{equation*}
is a... solution for the problem.

\section*{Problem 6}
WLOG, let $x\leq y$. I'll show that
\begin{equation*}
u(y)-u(x)=\int_y^x u' dt
\end{equation*}
Since $u'\in L^p(0, 1)$, $\abs{\int_x^y u' dt}\leq \norm{u'}_{L^p(\Omega)}$, and $u'\in L^1(\Omega)$.

Fix small enough $\epsilon>0$ and consider $\phi_\epsilon(t)\in C_c^\infty((x, y))$ such that $\phi_\epsilon\equiv 1$ in $[x+\epsilon, y-\epsilon]$. It can be generated using $C^\infty$ Urysohn's lemma. Then,
\begin{equation*}
\begin{split}
\int_\Omega u\phi_\epsilon'dt&=\int_{x}^{x+\epsilon} u \phi_\epsilon' dt + \int_{y-\epsilon}^{y} u \phi_\epsilon' dt \\
&=\int_\Omega u'\phi_\epsilon dt=\int_{x}^{y} u'\phi_\epsilon dt.
\end{split}
\end{equation*}
Since ...$\phi$ is continuous function with compact support, $\abs{\int_{x-\epsilon}^{y+\epsilon} u'\phi dt} \leq \norm{\phi}_{L^\infty(\Omega)}\int_{x-\epsilon}^{y+\epsilon} \abs{u'} dt< \infty$, so by Lebesgue dominance convergence theorem, $\lim\limits_{\epsilon\rightarrow 0} \int_0^{x+\epsilon} u'\phi_\epsilon dt=\int_0^x u' dt$.

Let's consider $\int_0^\epsilon u\phi_\epsilon' dt$. We know that $\int_0^\epsilon \phi_\epsilon' dt=\phi_\epsilon(\epsilon)-\phi_\epsilon(0)=1$. $\int_0^\epsilon u\phi_\epsilon' dt=\int_0^\epsilon (u-u(x))\phi_\epsilon' dt + \int_0^\epsilon u(x)\phi_\epsilon' dt$
\end{document}