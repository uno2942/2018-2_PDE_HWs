\documentclass{article}
\usepackage{graphicx, amssymb}
\usepackage{amsmath}
\usepackage{amsfonts}
\usepackage{amsthm}
\usepackage{kotex}
\usepackage{bm}
\usepackage{hyperref}
\usepackage{xcolor}
\usepackage{mathrsfs}
\usepackage{mathtools}
\usepackage{physics}
\usepackage{ esint }

\textwidth 6.5 truein 
\oddsidemargin 0 truein 
\evensidemargin -0.50 truein 
\topmargin -.5 truein 
\textheight 8.5in

\DeclareMathOperator{\cc}{\mathbb{C}}
\DeclareMathOperator{\rr}{\mathbb{R}}
\DeclareMathOperator{\bA}{\mathbb{A}}
\DeclareMathOperator{\fra}{\mathfrak{a}}
\DeclareMathOperator{\frb}{\mathfrak{b}}
\DeclareMathOperator{\frm}{\mathfrak{m}}
\DeclareMathOperator{\frp}{\mathfrak{p}}
\DeclareMathOperator{\slin}{\mathfrak{sl}}
\DeclareMathOperator{\Lie}{\mathsf{Lie}}
\DeclareMathOperator{\Alg}{\mathsf{Alg}}
\DeclareMathOperator{\Spec}{\mathrm{Spec}}
\DeclareMathOperator{\End}{\mathrm{End}}
\DeclareMathOperator{\rad}{\mathrm{rad}}
\newcommand*\Laplace{\mathop{}\!\mathbin\bigtriangleup}
\newcommand{\id}{\mathrm{id}}
\newcommand{\Hom}{\mathrm{Hom}}
\newcommand{\Sch}{\mathbf{Sch}}
\newcommand{\Ring}{\mathbf{Ring}}
\newcommand{\T}{\mathcal{T}}
\newcommand{\B}{\mathcal{B}}
\newcommand{\Mod}[1]{\ (\mathrm{mod}\ #1)}
\newtheorem{lemma}{Lemma}
\newtheorem{theorem}{Theorem}
\newtheorem{proposition}{Proposition}

\begin{document}


\title{Partial Differential Equation - HW 2}
\author{SungBin Park, 20150462} 

 \maketitle

\section*{Problem 1}
\begin{enumerate}
	\item [Step 1.] Let a linear transformation from $\rr^{n+1}$ to $\rr^{n+1}$ sending $s\mapsto r^2 s$ and $y_i\mapsto ry_i$, then $T$ sends $E(0, \bm{0}, r)$ to $E(0, \bm{0}, 1)$. Then
	\begin{equation*}
	\begin{split}
	\frac{1}{4}\iint_{E(0, \bm{0}, 1)}\frac{\abs{y}^2}{s^2}dyds&=\frac{1}{4}\iint_{E(0, \bm{0}, r)}\frac{\abs{y}^2/r^2}{s^2/r^4}\frac{1}{r^{n+2}}dyds \\
	&=\frac{1}{4r^n}\iint_{E(0, \bm{0}, r)}\frac{\abs{y}^2}{s^2}dyds
	\end{split}
	\end{equation*}
    \item [Step 2.] Since $e^x\leq 1$ if $x\leq 0$,
    \begin{equation*}
        \begin{split}
            &\frac{1}{\left(4\pi(-s)\right)^{n/2}}e^{\frac{\abs{y}^2}{4s}}\geq 1 \text{ for }s\leq 0\\
            &\Leftrightarrow 4\pi(-s)\leq 1 \text{ and }e^{\frac{\abs{y}^2}{4s}}\geq \left(4\pi(-s)\right)^{n/2} \text{ and } s\leq 0\\
            &\Leftrightarrow -\frac{1}{4\pi}\leq s\leq 0 \text{ and } \abs{y}^2\leq 2sn\ln{(-4\pi s)}
        \end{split}
    \end{equation*}
    Therefore, $E(0,0,1)=\left\{(s,y)\in R^{n+1}:-\frac{1}{4\pi}\leq s\leq 0 , \abs{y}^2\leq -2sn\ln{\left(\frac{1}{-4\pi s}\right)}\right\}$
    
    \item[Step 3.] Since $\abs{y}^2\leq -2sn\ln\left(\frac{1}{-4\pi s}\right)$ is the ball with radius $\left(-2sn\ln\left(\frac{1}{-4\pi s}\right)\right)^{1/2}$, 
    \begin{equation*}
    \begin{split}
        \iint_{E(0, \bm{0}, 1)} \frac{\abs{y}^2}{s^2}dyds&=\int_{-1/4\pi}^0\int_{B_s}\frac{\abs{y}^2}{s^2}dyds \text{ (By Fubini theorem)}\\
        &=\int_{-1/4\pi}^0 \frac{1}{s^2} \int_{S^{n-1}}\int_0^{\left(-2sn\ln\left(\frac{1}{-4\pi s}\right)\right)^{1/2}} r^{n+1}drd\sigma ds \text{ (By Spherical Coord.)}\\
        &=\int_{-1/4\pi}^0 \frac{1}{s^2}\frac{\frac{n}{2}\pi^{n/2}}{\Gamma(\frac{n}{2}+2)} \left(-2sn\ln\left(\frac{1}{-4\pi s}\right)\right)^{n/2+1} ds
    \end{split}
    \end{equation*}
    Let $t=\ln\left(\frac{1}{-4\pi s}\right)$, then 
    \begin{equation*}
        \begin{split}
            &\frac{1}{4}\left(\frac{n}{2\pi}\right)^{n/2+2}\frac{\pi^{n/2}}{\Gamma(n/2+2)}\int_{0}^\infty \frac{(4\pi)^2}{e^{-2t}} \left(e^{-t}t\right)^{n/2+1} e^{-t}dt \\
            &=4\left(\frac{n}{2}\right)^{n/2+2}\frac{1}{\Gamma(n/2+2)}\int_0^\infty t^{n/2+1}e^{-(n/2)t}dt \\
            &=4\left(\frac{n}{2}\right)^{n/2+2}\frac{1}{\Gamma(n/2+2)}\left(\frac{2}{n}\right)^{n/2+2}\Gamma(n/2+2)=4.
        \end{split}
    \end{equation*}
    Therefore, $\frac{1}{4}\iint_{E(0, \bm{0}, 1)} \frac{\abs{y}^2}{s^2}dyds=1$.
\end{enumerate}
Summarising the results, $\frac{1}{4r^n}\iint_{E(0, \bm{0}, r)}\frac{\abs{y}^2}{s^2}dydx=1$ for all $r>0$.

\section*{Problem 2}
By change of variable shifting $x$ and $t$ to 0, we can rearrange the formula by 
\begin{equation*}
\begin{split}
\frac{1}{r^n}\iint_{E(t,x;r)}u(s,y)\frac{\abs{s-y}^2}{\abs{t-s}^2}dyds&=\frac{1}{r^n}\iint_{E(0,\bm{0};r)}u'(s,y)\frac{\abs{y}^2}{s^2}dyds\\
&=\iint_{E(0,\bm{0};1)}u'(r^2s',ry')\frac{\abs{y'}^2}{s'^2}dy'ds'
\end{split}
\end{equation*}
where $u'(s,y)=u(s+t, y+x)$ and $s=r^2s'$, $y=ry'$. For simplicity, I'll write $u$ by $u'$
.
Let $\phi(r)$ be the RHS of above equation. Our strategy is showing that $\phi(r)$ is constant function, and it means $\phi(r)=\lim\limits_{r\rightarrow 0} \phi(r)=r^n u(0,0)$ using continuity of $u$ and the result from problem 1. I'll write $E(r)=E(0,\bm{0};r)$.
Computing $\phi'(r)$,
\begin{equation*}
\begin{split}
\phi'(r)&=\dv{r}\iint_{E(1)} u(r^2s',ry')\frac{\abs{y'}^2}{s'^2}dy'ds' \\
&= \iint_{E(1)}(2rs'u_s(r^2s', ry')+y'\cdot \nabla_y u(r^2s', ry'))\frac{\abs{y'}^2}{s'^2}dy'ds' \text { (Since }\frac{\abs{y}^2}{s^2}\in L^1(E(1)) \text{ and } u\in C(\overline{E(1)}) \\
&= \frac{1}{r^{n+1}}\iint_{E(r)}(2su_s(s,y)+y\cdot \nabla_y u)\frac{\abs{y}^2}{s^2}dyds
\end{split}
\end{equation*}
Let the first term $A$ and last term $B$.

For simple calculation, I'll introduce a function
\begin{equation*}
\varphi(s, y)= -\frac{n}{2}\ln(-4\pi s)+\frac{\abs{y}^2}{4s}+n\ln r
\end{equation*}
Then, we can get a relation $\abs{y}^2=\sum\limits_{i=1}^n 2sy_i\varphi_{y_i}$, and
\begin{equation*}
\begin{split}
r^{n+1}A&=\iint_{E(r)} 4u_s \sum\limits_{i=1}^n y_i \varphi_{y_i} dyds \\
&=\iint_{E(r)} 4 \sum\limits_{i=1}^n \left(\pdv{y_i}\left(y_iu_s\varphi\right)-u_s\varphi-u_{sy_i}y_i\varphi\right) dyds \\
&=-4\iint_{E(r)} nu_s\varphi + \varphi\sum\limits_{i=1}^n u_{sy_i} y_i  dyds
\end{split}
\end{equation*}
Since
\begin{equation*}
\begin{split}
\iint_{E(r)} \sum\limits_{i=1}^n \pdv{y_i}\left(y_i u_s\varphi\right) dyds&=\sum\limits_{i=1}^n \int_{-\frac{r^2}{4\pi}}^0 \int_{B\left(\bm{0}, -2ns \ln{\left(-\frac{4\pi s}{r^2}\right)}\right)}  \pdv{y_i}\left(y_i u_s\varphi\right) dyds \\
&=\sum\limits_{i=1}^n \int_{-\frac{r^2}{4\pi}}^0 \int_{\partial B\left(\bm{0}, -2ns \ln{\left(-\frac{4\pi s}{r^2}\right)}\right)}  \left(y_i u_s\varphi\right)\nu^i dyds \\
&=0
\end{split}
\end{equation*}
for $\varphi\equiv 0$ in $\partial E-\{0,\bm{0}\}$.($\nu^i$ is $i$th component of outward pointing unit normal vector.)

By $\pdv{s}\left(u_{y_i}y_i\varphi\right)=u_{sy_i}y_i\varphi+u_{y_i}\varphi y_i \varphi_s$ 
\begin{equation*}
\begin{split}
\iint_{E(r)} \varphi\sum\limits_{i=1}^n u_{sy_i} y_i  dyds&=\iint_{E(r)} \sum\limits_{i=1}^n \pdv{s}\left(u_{y_i}y_i\varphi\right) - u_{y_i}y_i \varphi_s dyds
\end{split}
\end{equation*}
Let $s_0(y)$ and $s_1(y)$ is the the boundary points of $s$ for fixed $y\neq \bm{0}$, and $\{y\}\times(s_0(y),s_1(y))$ is in the heat ball. In the integration on $E(r)$, we can neglect $y=0$ case since it $\{\bm{0}\}\times(-1/4\pi, 0)$ is measure zero set in $\Omega$. Then,
\begin{equation*}
\begin{split}
\iint_{E(r)}\pdv{s}\left(u_{y_i}y_i\varphi\right)dyds&=\int_{\abs{y}^2\leq\frac{nr^2}{2\pi e}}\int_{s_0(y)}^{s_1(y)}\pdv{s}\left(u_{y_i}y_i\varphi\right)dsdy \\
&=\int_{\abs{y}^2\leq\frac{nr^2}{2\pi e}}u_{y_i}(s_1(y), y), y)y_i\varphi(s_1(y))-u_{y_i}(s_0(y), y), y)y_i\varphi(s_0(y))dy=0.
\end{split}
\end{equation*}
because $\varphi(s_0(y))=\varphi(s_1(y))=0$ for $y\neq \bm{0}$.

Therefore,
\begin{equation*}
\begin{split}
r^{n+1}A=-4\iint_{E(r)} nu_s\varphi + \varphi\sum\limits_{i=1}^n u_{sy_i} y_i  dyds &= -4\iint_{E(r)} nu_s\varphi - \varphi\sum\limits_{i=1}^n u_{y_i} y_i \varphi_s dyds \\
&=-4\iint_{E(r)} nu_s\varphi - \sum\limits_{i=1}^n u_{y_i} y_i \left(-\frac{n}{2s}-\frac{\abs{y}^2}{4s^2}\right) dyds \\
&=\iint_{E(r)} -4nu_s\varphi - \frac{2n}{s}\sum\limits_{i=1}^n u_{y_i} y_i dyds - B.
\end{split}
\end{equation*}
Since $u$ is a solution for heat equation,
\begin{equation*}
\begin{split}
\phi'(r)=A+B&=\iint_{E(r)} -4nu_s\varphi - \frac{2n}{s}\sum\limits_{i=1}^n u_{y_i} y_i dyds \\
&=\iint_{E(r)} -4n\varphi\Laplace_y u - \frac{2n}{s}\sum\limits_{i=1}^n u_{y_i} y_i dyds \\
&=\iint_{E(r)} \sum\limits_{i=1}^n-4n u_{y_i}\varphi_{y_i} - \frac{2n}{s}\sum\limits_{i=1}^n u_{y_i} y_i dyds \\
&=\iint_{E(r)} \sum\limits_{i=1}^n\frac{2n}{s} u_{y_i}y_i - \frac{2n}{s}\sum\limits_{i=1}^n u_{y_i} y_i dyds = 0.
\end{split}
\end{equation*}
The intermediate step $\varphi\Laplace_y u \rightarrow u_{y_i}\varphi_{y_i}$ uses the fact that $\varphi=0$ on $\partial E(r)-\{(0, \bm{0}\}$ as in $s$ case.

Consequently, $\phi$ is a constant function. Since $u$ is continuous and by problem 1,
\begin{equation*}
\lim\limits_{r\rightarrow 0+}\frac{1}{r^n}\iint_{E(0,\bm{0};r)}u(s,y)\frac{\abs{y}^2}{s^2}dyds =4u(0,0) 
\end{equation*}
Hence, 
\begin{equation*}
\frac{1}{4r^n}\iint_{E(t,x;r)}u(s,y)\frac{\abs{s-y}^2}{\abs{t-s}^2}dyds = u(t,x)
\end{equation*}
for all $r>0$.
\section*{Problem 3}
\begin{enumerate}
\item[Step 1.] First, we need to assume that the convergence radius of $u$ is $\infty$, so that we can safely differentiate the series term by term. To satisfy $u_t-u_{xx}=0$ for $t>0, x\in \rr$,
\begin{equation*}
u_t-u_{xx}=\sum_{j=0}^\infty \left(\pdv{g_j(t)}{t}-(j+1)(j+2)g_{j+2}(t)\right)x^{j}=0.
\end{equation*}
By uniqueness of power series in the convergence region, $\pdv{g_j(t)}{t}-(j+1)(j+2)g_{j+2}=0$ for all $j\geq 0$ iff $u_t-u_{xx}=0$.

To satisfy initial condition $u(0, x)=0$ for $x\in \rr$, $\sum_{j=0}^\infty g_j(0)x^j=0$, $g_j(0)$ should be $0$ for all $j$ followed by uniqueness of power series.

\item[Step 2.] I'll use induction for proof. Let $g_j$ be
\begin{equation}\label{Eq:3.1}
g_j(t)=\begin{cases}
0 & j\text{ is odd.} \\
\frac{1}{(2(j/2))!}\frac{d^{(j/2)}}{dt^{(j/2)}}g(t) & j\text{ is even.}
\end{cases}
\end{equation}
for $j<2n$. The starting case is given in the problem. For $j=2n$,
\begin{equation*}
g_{2n}=\frac{1}{(2n-1)(2n)}\pdv{g_{2n-2}(t)}{t}=\frac{1}{2n!}\frac{d^{n}}{dt^{n}}g(t).
\end{equation*}
For $j=2n+1$,
\begin{equation*}
g_{2n+1}=\frac{1}{(2n)(2n+1)}\pdv{g_{2n-1}(t)}{t}=0.
\end{equation*}
Therefore, \eqref{Eq:3.1} is valid.
\item[Step 3.] Since $\frac{1}{z^2}$ is holomorphic without $z= 0$, $e^{-\frac{1}{z^2}}$ is holomorphic without $z=0$.
For $t> 0$, let $r=t/8$ be a radius of open disk centred at $t\in \rr$ in $\mathbb{C}$. Let the boundary of the disk $C$. Then by Cauchy integral formula,
\begin{equation*}
\abs{\frac{d^k}{dt^k}g(t)}\leq \frac{k!}{2\pi}\varointctrclockwise_C \abs{\frac{e^{-\frac{1}{z^2}}}{(z-t)^{k+1}}}dz\leq \frac{k!}{2\pi} r^{-k} \max\limits_C\abs{e^{-\frac{1}{z^2}}}
\end{equation*}
and if we let $z=x+iy$, $x,y\in \rr$,
\begin{equation*}
\abs{e^{-\frac{1}{z^2}}}=\abs{e^{-\frac{x^2-y^2}{(x^2+y^2)^2}}}\leq \abs{e^{-\frac{1280}{1681}t^{-2}}}\leq\abs{e^{-\frac{1}{2t^2}}}.
\end{equation*}
Therefore, if we set $\theta=\frac{1}{8}$,
\begin{equation*}
\abs{\frac{d^k}{dt^k}g(t)}\leq k!\left(\frac{1}{8}t\right)^{-k}e^{-\frac{1}{2t^2}}
\end{equation*}
If $t\leq 0$, it is obvious inequality since $g^(k)(t)=0$ for $t<0$ and $\lim\limits_{t\rightarrow 0} g(k)(t)=0$....
\item[Step 4.] Let $g_i$ as in Step 2. Then,
\begin{equation*}
\abs{g_{2k}(t)x^{2k}}\leq \frac{k!}{(2k)!}e^{-\frac{1}{2t^2}}\left(\frac{8x^2}{t}\right)^k.
\end{equation*}
The radius of convergence of RHS is
\begin{equation*}
R^2\leq \lim\limits_{k\rightarrow \infty}(2k+1)\frac{t}{4},
\end{equation*}
and it means $\sum g_{2k}(t)x^{2k}$ converges for all $t>0$ and $x\in \rr$.

Since $\frac{k!}{(2k)!}e^{-\frac{1}{2t^2}}\left(\frac{8x^2}{t}\right)^k\leq \frac{1}{k!}e^{-\frac{1}{2t^2}}\left(\frac{8x^2}{t}\right)^k$, by replacing $\frac{8x^2}{t}$ by $z$, we can get
\begin{equation*}
\abs{\sum g_{2k}(t)x^{2k}}\leq e^{-\frac{1}{2t^2}}\sum\frac{1}{k!}z^k=e^{-\frac{1}{2t^2}+\frac{8x^2}{t}}
\end{equation*}
for fixed $t>0$ with convergence of radius $\infty$. As $t\rightarrow 0$, $e^{-\frac{1}{2t^2}+\frac{8x^2}{t}}\rightarrow 0$. Therefore, 
\begin{equation*}
\sum g_{2k}(t)x^{2k}\rightarrow 0 ~\text{ as }t\rightarrow 0.
\end{equation*}
\end{enumerate}

\section*{Problem 4}
Consider the following PDE:
\begin{equation}\label{Eq:P4-1}
u_t-\Laplace_x u=f(t,x)e^{ct}~~\text{for }t>0,x\in\rr^n,~~u(0,x)=g(x)
\end{equation}
Then, $f(t,x)e^{ct}$ and $g(x)$ are smooth and have compact supports. Therefore, there exists a solution $u$ satisfying \eqref{Eq:P4-1}:
\begin{equation*}
u(x,y)=\int_{\rr^n}\Phi(x-y,t)g(y)dy+\int_0^t\int_{\rr^n}\Phi(x-y,t-s)f(y,s)e^{ct}dyds
\end{equation*}
Let's consider $u'=ue^{-ct}$, then
\begin{equation*}
\begin{split}
&u'_t-\Laplace_x u'+cu'=u_te^{-ct}-cue^{ct}-\Laplace_x ue^{ct}+cue^{ct}=(u_t-\Laplace_x u)e^{ct}=f(t,x) \\
&u'(0,x)=ue^{c\cdot 0}=u=g(x)
\end{split}
\end{equation*}
for $t>0$ and $x\in \rr^n$. Therefore, 
\begin{equation*}
e^{-ct}\left(\int_{\rr^n}\Phi(x-y,t)g(y)dy+\int_0^t\int_{\rr^n}\Phi(x-y,t-s)f(y,s)e^{ct}dyds\right)
\end{equation*}
is a solution for the problem.

\section*{Problem 5}
Let $u$ be:
\begin{equation}\label{Eq:P5-1}
u(t,x)=\frac{1}{2}[g(x+t)+g(x-t)]+\frac{1}{2}\int_{x-t}^{x+t} h(y)dy + \frac{1}{2}\int_0^t\int_{x+t-s}^{x-t+s}f(y, s)dyds.
\end{equation}
First, $u$ is a smooth function since $g,h,f$ are. Also,
\begin{equation*}
\begin{split}
u_t&=\frac{1}{2}\left(g'(x+t)-g'(x-t)\right)+\left(h(x+t)+h(x-t)\right)+\frac{1}{2}\int_0^t f(x+t-s, s)+f(x-t+s, s)ds\\
u_{tt}&=\frac{1}{2}\left(g''(x+t)+g''(x-t)\right)+\left(h'(x+t)-h'(x-t)\right)+f(x)+\int_0^t f_t(x+t-s)+f_t(x-t+s)ds\\
u_x&=\frac{1}{2}\left(g'(x+t)+g'(x-t)\right)+\left(h(x+t)-h(x-t)\right)+\frac{1}{2}\int_0^t f(x+t-s, s)-f(x-t+s, s)ds \\
u_{xx}&=\frac{1}{2}\left(g''(x+t)+g''(x-t)\right)+\left(h'(x+t)-h'(x-t)\right)+\frac{1}{2}\int_0^t f_x(x+t-s, s)-f_x(x-t+s, s)ds \\
&=\frac{1}{2}\left(g''(x+t)+g''(x-t)\right)+\left(h'(x+t)-h'(x-t)\right)+\frac{1}{2}\int_0^t f_t(x+t-s, s)+f_t(x-t+s, s)ds \\
&=u_{tt}-f(x,t)
\end{split}
\end{equation*}
Therefore, $u_{tt}-u_{xx}=f(x,t)$.

Also,
\begin{equation*}
\lim_{\mathclap{\substack{(x,t)\rightarrow (x_0, 0)\\ t>0}}} u(t,x)=g(x_0, 0),~~\lim_{\mathclap{\substack{(x,t)\rightarrow (x_0, 0)\\ t>0}}} u_t(x,t)=h(x_0, 0)
\end{equation*}
since $u_t$ and $u$ is continuous on $\rr^2$.

Hence, \eqref{Eq:P5-1} is a solution for the problem.
\section*{Problem 6}
WLOG, let $x\leq y$. I'll show that
\begin{equation*}
u(y)-u(x)=\int_y^x u' dt
\end{equation*}
Since $u'\in L^p(0, 1)$, $\int_x^y \abs{u'} dt\leq \norm{u'}_{L^p(\Omega)}$, and $u'\in L^1(\Omega)$.

If $x=y$, it is trivially equal, so I'll set $x<y$. Fix small enough $0<\epsilon<\frac{x+y}{2}$ and consider $\phi_\epsilon(t)\in C_c^\infty((x, y))$ such that $\phi_\epsilon\equiv 1$ in $[x+\epsilon, y-\epsilon]$. Define 
\begin{equation*}
\eta_\delta(x)=
\begin{cases}
\frac{1}{\delta}e^{\frac{1}{\frac{x^2}{\delta^2}-1}} & \text{(For }\abs{x}<\delta\text{)} \\
0 & \text{(For }\abs{x}\geq\delta\text{)}
\end{cases}
\end{equation*}
I'll set the $\phi_\epsilon(t)$ by
\begin{equation*}
\phi_\epsilon(t)=
\begin{cases}
\int_x^t \eta_\epsilon(\sigma-x)d\sigma/\int_x^{x+\epsilon} \eta_\epsilon(\sigma-x)d\sigma & \text{(For }x\leq t<\frac{x+y}{2}\text{)} \\
1-\left(\int_{y-\epsilon}^t \eta_\epsilon(\sigma-(y-\epsilon))d\sigma/\int_{y-\epsilon}^y \eta_\epsilon(\sigma-(y-\epsilon))d\sigma\right) & \text{(For }\frac{x+y}{2}\leq t\leq y\text{)}
\end{cases}
\end{equation*}
Then,
\begin{equation*}
\begin{split}
\int_\Omega u\phi_\epsilon'dt&=\int_{x}^{x+\epsilon} u \phi_\epsilon' dt + \int_{y-\epsilon}^{y} u \phi_\epsilon' dt \\
&=-\int_\Omega u'\phi_\epsilon dt=-\int_{x}^{y} u'\phi_\epsilon dt.
\end{split}
\end{equation*}

Since $u'\in L^1(\Omega)$, $\abs{u'\phi_\epsilon}\leq \abs{u'}$, and $\phi_\epsilon\rightarrow 1$ in $[x,y]$ a.e., so by Lebesgue dominance convergence theorem, $\lim\limits_{\epsilon\rightarrow 0} \int_x^y u'\phi_\epsilon dt=\int_x^y u' dt$.

Let's consider $\int_x^{x+\epsilon} u\phi_\epsilon' dt$. We know that $\int_x^{x+\epsilon} \phi_\epsilon' dt=\phi_\epsilon(x+\epsilon)-\phi_\epsilon(x)=1$ and $\int_0^\epsilon \eta_\epsilon(\sigma)\sigma=\int_0^1 \eta(\sigma)\sigma=C>0$. Therefore,
\begin{equation*}
\begin{split}
\int_x^{x+\epsilon} u\phi_\epsilon' dt&=C^{-1}\int_x^{x+\epsilon} (u - u(x)) \eta_\epsilon(t-x) dt + \int_0^\epsilon u(x)\phi_\epsilon' dt \\
&=\int_x^{x+\epsilon} (u - u(x)) \eta_\epsilon(t-x) dt+u(x).
\end{split}
\end{equation*}
Since $\norm{\eta_\epsilon}_{L^\infty}=\frac{1}{\epsilon}$ and $C^{-1}\abs{\int_x^{x+\epsilon} (u - u(x)) \eta_\epsilon(t-x) dt} \leq \frac{\norm{\eta_\epsilon}_{L^\infty}}{C}\abs{\frac{1}{\epsilon}\int_x^{x+\epsilon} (u - u(x)) dt}$, by Lebesgue differential theorem, $\int_x^{x+\epsilon} (u - u(x)) \eta_\epsilon(t-x) dt\rightarrow 0$ as $\epsilon\rightarrow 0$. 
Therefore, $\int_x^{x+\epsilon} u\phi_\epsilon' dt \rightarrow u(x)$ as $\epsilon\rightarrow 0$. 
By the same reason, $\int_{y-\epsilon}^y u\phi_\epsilon' dt \rightarrow -u(y)$ as $\epsilon\rightarrow 0$.

Therefore,
\begin{equation*}
\int_x^y u' dt = u(y) - u(x)
\end{equation*}
a.e. and
\begin{equation*}
\abs{u(y)-u(x)}\leq \abs{\int_x^y u' dt} \leq \abs{x-y}^{1-\frac{1}{p}}\norm{u'}_{L^p(\Omega)}
\end{equation*}
by H\"older's inequality.
\section*{Problem 7}
Because $u$ is compactly supported, we can set $B$ be a large ball containing the compact support of $u$ and set $u=0$ out of compact support for integration; it does not effect the integration. Since $p\geq 2$, we can do integration by parts:
\begin{equation*}
\begin{split}
\int_U \abs{Du}^p dx&=\sum\limits_{i=1}^n \int_B u_{x_i} u_{x_i} \abs{Du}^{p-2}dx \\
&=-\sum\limits_{i,j=1}^n \int_B u \left(u_{x_ix_i}\abs{Du}^{p-2}+u_{x_i}u_{x_j}u_{x_jx_i}\abs{Du}^{p-4}\right)dx ~\text{ Since }u\equiv 0\text{ at boundary}\\
&\leq-\sum\limits_{i,j=1}^n \int_B u \left(u_{x_ix_i}\abs{Du}^{p-2}+Bu_{x_jx_i}\abs{Du}^{p-2}\right)dx~\text{ Since }\sum_{i,j} u_{x_i}u_{x_j}\leq n\abs{Du^2}\text{ by Cauchy-Schwarz inequality} \\
&\leq -C\int_B u \abs{D^2u}\left(\abs{Du}^{p-2}\right)dx~\text{ By the same reason}. \\
&\leq \abs{C\int_U u \abs{D^2u}\left(\abs{Du}^{p-2}\right)dx}
\end{split}
\end{equation*}
for some constant $B$ and $C$ depends on $n$.

Let $p>2$, then by the H\"older's inequality, 
\begin{equation*}
\abs{\int_U u \abs{D^2u}\left(\abs{Du}^{p-2}\right)dx}^p\leq \left(\int_U \abs{u\abs{D^2u}}^{\frac{p}{2}} dx\right)^2\left(\int_U \abs{Du}^{p} dx\right)^{p-2}
\end{equation*}
If $\int_U \abs{Du}^{p} dx=0$, then the original inequality satisfied, so we can assume $\int_U \abs{Du}^{p} dx>0$. Then,
\begin{equation*}
\left(\int_U \abs{Du}^{p} dx\right)^2\leq C^p\left(\int_U \abs{u\abs{D^2u}}^{\frac{p}{2}} dx\right)^2\leq C^p\left(\int_U \abs{u}^p dx\right) \left(\int_U \abs{D^2 u}^p dx\right)
\end{equation*}
Therefore, 
\begin{equation*}
\norm{Du}_{L^p}\leq C\norm{u}_{L^p}^{1/2}\norm{D^u}_{L^p}^{1/2}
\end{equation*}
for $2\leq p\leq \infty $ and all $u\in C^\infty_c(U)$.
\end{document}