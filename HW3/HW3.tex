\documentclass{article}
\usepackage{graphicx, amssymb}
\usepackage{amsmath}
\usepackage{amsfonts}
\usepackage{amsthm}
\usepackage{kotex}
\usepackage{bm}
\usepackage{hyperref}
\usepackage{xcolor}
\usepackage{mathrsfs}
\usepackage{mathtools}
\usepackage{physics}
\usepackage{ esint }

\textwidth 6.5 truein 
\oddsidemargin 0 truein 
\evensidemargin -0.50 truein 
\topmargin -.5 truein 
\textheight 8.5in

\DeclareMathOperator{\cc}{\mathbb{C}}
\DeclareMathOperator{\rr}{\mathbb{R}}
\DeclareMathOperator{\bA}{\mathbb{A}}
\DeclareMathOperator{\fra}{\mathfrak{a}}
\DeclareMathOperator{\frb}{\mathfrak{b}}
\DeclareMathOperator{\frm}{\mathfrak{m}}
\DeclareMathOperator{\frp}{\mathfrak{p}}
\DeclareMathOperator{\slin}{\mathfrak{sl}}
\DeclareMathOperator{\Lie}{\mathsf{Lie}}
\DeclareMathOperator{\Alg}{\mathsf{Alg}}
\DeclareMathOperator{\Spec}{\mathrm{Spec}}
\DeclareMathOperator{\End}{\mathrm{End}}
\DeclareMathOperator{\rad}{\mathrm{rad}}
\newcommand*\Laplace{\mathop{}\!\mathbin\bigtriangleup}
\newcommand{\id}{\mathrm{id}}
\newcommand{\Hom}{\mathrm{Hom}}
\newcommand{\Sch}{\mathbf{Sch}}
\newcommand{\Ring}{\mathbf{Ring}}
\newcommand{\T}{\mathcal{T}}
\newcommand{\B}{\mathcal{B}}
\newcommand{\Mod}[1]{\ (\mathrm{mod}\ #1)}
\newtheorem{lemma}{Lemma}
\newtheorem{theorem}{Theorem}
\newtheorem{proposition}{Proposition}

\begin{document}


\title{Partial Differential Equation - HW 3}
\author{SungBin Park, 20150462} 

 \maketitle

\section*{Problem 1}
I'll use the proof in Evans.
\begin{proof}
Before starting, let's arrange the index of $\Gamma_j$ so that the adjacent curve of $\Gamma_j$ be $\Gamma_{j-1}$ and $\Gamma_{j+1}$.

Fix $x^0$ be in end point of $\Gamma_j$ and assume that $x^0$ is also a end point of $\Gamma_{j+1}$. Let $v_{j}$ be the tangential vector of $\Gamma_j$ such that $v_j$ is toward the $\Gamma_j$. In other words, if $\Gamma_j:[a,b]\rightarrow \rr^2$ and $x_0=\Gamma_j(b)$, then $v_j=\lim\limits_{h\rightarrow 0+}\frac{\Gamma_j(b-h)-\Gamma_j(b)}{h}$. For $v_{j+1}$, set it be tangential vector of $\Gamma_{j+1}$ toward $\Gamma_{j+1}$ curve. If $v_{j}$ and $v_{j+1}$ are parallel, $\Gamma_j$ and $\Gamma_{j+1}$ can be connected with $C^1$ property, I'll ignore the case. Let the angle between $v_j$ and $v_{j+1}$ be $\theta>0$. Let $e_0$ be a unit vector such that parallel with $\frac{v_j+v_{j+1}}{2}$ and inward direction, i.e., $x_0+\lambda e_0\in \text{int }\Omega$ for small enough $\lambda$.(This requires Jordan curve theorem.)

As $\Gamma_j$, $\Gamma_{j+1}$ are $C^1$, there exists $r>0$ such that with $B(x_0, r)$, the tangential direction of $\Gamma_j$ and $\Gamma_{j+1}$ does not change much. More precisely, if we let $w_j$ be a tangential vector of $\Gamma_j$ at $y\in B(x_0, r)$, then the angle between $w_j$ and $v_j$ is less than $\theta/10$, and this is true for $\Gamma_{j+1}$. Let $w^1_j$ (resp. $w^2_j$) be the vector made by rotating $v_j$ by $\theta/10$ clockwise (resp. counter-clockwise). Do the same for $w^1_{j+1}$, $w^2_{j+1}$.

Now, let's repeat proof in Evans. Let's consider $U\cap B(x_0, r)$ and $V\coloneqq U\cap B(x_0, r/2)$. Define $x_\epsilon\coloneqq x+\epsilon e_0$ for $x\in V$, small enough $\epsilon>0$ satisfying $x+\epsilon e_0\in U\cap B(x_0, r)$. WLOG, I'll assume that $x$ is inside the interior enclosed by $\Gamma_j$ and the line through $x_0$ with tangential vector $e_0$. Now, we draw lines through $x$ such that the tangential vectors $w^1_j$ and $w^2_j$. Then, we know that the angle between $e_0$ and $w^1_j$ or $w^2_j$ is $(1/2-1/10)\theta$ and there is a room to set small enough $\lambda<1$ such that $B(x+\epsilon e_0, \lambda \epsilon)\subset U\cap B(x_0, r)$. In this room, we can mollify $u_\epsilon(x)=u(x_\epsilon)$ and denote it $v_\epsilon$, and make $\epsilon\rightarrow 0$. The remaining part is same as the proof in Evans: Since $\partial U$ is compact, we can choose finitely many points $x_0^i\in \partial U$ including end point of $\Gamma_j$ and make Global approximation.
\end{proof}
\section*{Problem 2}
\begin{enumerate}
\item[1.] $W_0^{1,p}(\Omega)$ is a vector space: For $f=0$, $f\in W_0^{1,p}(\Omega)$, so $W_0^{1,p}(\Omega)\neq \phi$. For $f_1, f_2\in W_0^{1,p}(\Omega)$, there exists $f_1^j, f_2^j$ such that $(f_1^j), (f_2^j)\in C_c^\infty (\Omega)$ and $f_1^j\rightarrow f_1$, $f_2^j\rightarrow f_2$ in $W^{1,p}(\Omega)$. Since union of two compact set in $\Omega$ is compact in $\Omega$, $f_1^j+f_2^j\in C_c^\infty(\Omega)$ and for large enough $N$ satisfying $\norm{f_1^j-f_1}_{W^{1,p}(\Omega)} , \norm{f_2^j-f_2}_{W^{1,p}(\Omega)}\leq \epsilon/2$ for $j>N$, $\norm{f_1^j+f_2^j-f_1-f_2}_{W^{1,p}(\Omega)}\leq \norm{f_1^j-f_1}_{W^{1,p}(\Omega)}+\norm{f_2^j-f_2}_{W^{1,p}(\Omega)}\leq \epsilon$. Therefore, $f_1^j+f_2^j\rightarrow f_1+f_2$ in $W^{1,p}(\Omega)$, so $f_1+f_2\in W_0^{1,p}(\Omega)$. Also, $\lambda f^j\rightarrow \lambda f$ in $W^{1,p}(\Omega)$ for scalar $\lambda$. Therefore, $W_0^{1,p}$ is vector space.
\item[2.] With the norm $\norm{\cdot}_{W^{1,p}(\Omega)}$, ${W_0^{1,p}(\Omega)}$ is Banach space: Let $f_j$ be a cauchy sequence in ${W_0^{1,p}(\Omega)}$. Since ${W^{1,p}(\Omega)}$ is Banach space, $f_j\rightarrow f$ in $W^{1,p}(\Omega)$. Since $\Omega$ is bounded and $\partial \Omega$ is $C^1$, there exists bounded linear operator $T:{W^{1,p}(\Omega)}\rightarrow L^p(\partial \Omega)$ and $Tf_j\equiv 0$ on $\partial U$ as $f_j\in {W_0^{1,p}(\Omega)}$. Then,
\begin{equation*}
\lim\limits_{j\rightarrow\infty}\norm{Tf_j-Tf}_{W^{1,p}(\Omega)}=\lim\limits_{j\rightarrow\infty}\norm{T(f_j-f)}_{W^{1,p}(\Omega)}\leq \lim\limits_{j\rightarrow\infty}\norm{T}_{W^{1,p}(\Omega)}\norm{f_j-f}_{W^{1,p}(\Omega)}=0
\end{equation*}
as $\norm{T}_{W^{1,p}(\Omega)}$ is bounded. Therefore, $Tf_j\rightarrow Tf$ in $W^{1,p}(\Omega)$ and $\lim\limits_{j\rightarrow \infty}\norm{Tf_j}_{W^{1,p}(\Omega)}=\norm{Tf}_{W^{1,p}(\Omega)}=0$. As a result, $f\in W_0^{1,p}(\Omega)$ and it implies Cauchy sequence in ${W_0^{1,p}(\Omega)}$ converges.
\end{enumerate}
Therefore, $W_0^{1,p}(\Omega)$ is Banach space.
\section*{Problem 3}
For $k\in \mathbb{N}$ and $\alpha\in(0,1]$,
\begin{equation*}
C^{k,\alpha}(\bar{\Omega})\coloneqq \{u\in C^{k}(\bar{\Omega}):\norm{u}_{C^{k,\alpha}(\bar{\Omega})}<\infty\}
\end{equation*}
Before starting, I need to show that $\norm{\cdot}_{C^{k,\alpha}(\bar{\Omega})}$ is a norm on ${C^{k,\alpha}(\bar{\Omega})}$.
\begin{proof}
\begin{enumerate}
\item[1.] By the definition of ${C^{k,\alpha}(\bar{\Omega})}$, we know that $\norm{u}_{C^{k,\alpha}(\bar{\Omega})}< \infty$ for any $u\in{C^{k,\alpha}(\bar{\Omega})}$. Let $u,v\in {C^{k,\alpha}(\bar{\Omega})}$. Then
\begin{equation*}
\begin{split}
\norm{u+v}_{C^{k,\alpha}(\bar{\Omega})}&=\sum\limits_{\abs{\alpha}\leq k}\norm{D^\alpha (u+v)}_{C(\bar{\Omega})}+\sum\limits_{\abs{\alpha}=k}\left[D^\alpha (u+v)\right]_{C^{0, \alpha}(\bar{\Omega})} \\
&=\sum\limits_{\abs{\alpha}\leq k}\sup_{x\in \Omega}\abs{D^\alpha (u+v)}+\sum\limits_{\abs{\alpha}=k}\sup_{\substack{x,y\in \Omega \\ x\neq y}} \left\{  \frac{\abs{D^\alpha (u+v)(x)-D^\alpha (u+v)(y)}}{\abs{x-y}^\alpha}\right\} \\
&\leq \sum\limits_{\abs{\alpha}\leq k}\sup_{x\in \Omega}\abs{D^\alpha u}+\sup_{x\in \Omega}\abs{D^\alpha v}+\sum\limits_{\abs{\alpha}=k}\sup_{\substack{x,y\in \Omega \\ x\neq y}} \left\{  \frac{\abs{D^\alpha u(x)-D^\alpha u(y)}+\abs{D^\alpha v(x)-D^\alpha v(y)}}{\abs{x-y}^\alpha}\right\} \\
&\leq \sum\limits_{\abs{\alpha}\leq k}\sup_{x\in \Omega}\abs{D^\alpha u}+\sup_{x\in \Omega}\abs{D^\alpha v}+\sum\limits_{\abs{\alpha}=k}\sup_{\substack{x,y\in \Omega \\ x\neq y}} \left\{  \frac{\abs{D^\alpha u(x)-D^\alpha u(y)}}{{\abs{x-y}^\alpha}}\right\}+\sup_{\substack{x,y\in \Omega \\ x\neq y}}\left\{  \frac{\abs{D^\alpha v(x)-D^\alpha v(y)}}{{\abs{x-y}^\alpha}}\right\} \\
&=\norm{u}_{C^{k,\alpha}(\bar{\Omega})}+\norm{v}_{C^{k,\alpha}(\bar{\Omega})}
\end{split}
\end{equation*}
Therefore, $\norm{u+v}_{C^{k,\alpha}(\bar{\Omega})}\leq \norm{u}_{C^{k,\alpha}(\bar{\Omega})}+\norm{v}_{C^{k,\alpha}(\bar{\Omega})}$.
\item[2.] For $\lambda\in \rr$,
\begin{equation*}
\begin{split}
\norm{\lambda u}_{C^{k,\alpha}(\bar{\Omega})}&=\sum\limits_{\abs{\alpha}\leq k}\sup_{x\in \Omega}\abs{D^\alpha \lambda u}+\sum\limits_{\abs{\alpha}=k}\sup_{\substack{x,y\in \Omega \\ x\neq y}} \left\{  \frac{\abs{D^\alpha \lambda u(x)-D^\alpha \lambda u(y)}}{\abs{x-y}^\alpha}\right\} \\
&=\abs{\lambda}\sum\limits_{\abs{\alpha}\leq k}\sup_{x\in \Omega}\abs{D^\alpha  u}+\abs{\lambda}\sum\limits_{\abs{\alpha}=k}\sup_{\substack{x,y\in \Omega \\ x\neq y}} \left\{  \frac{\abs{D^\alpha  u(x)-D^\alpha  u(y)}}{\abs{x-y}^\alpha}\right\}\\
&=\lambda\norm{u}_{C^{k,\alpha}(\bar{\Omega})}.
\end{split}
\end{equation*}
\item[3.] For $u=0$, $\norm{u}_{C^{k,\alpha}(\bar{\Omega})}=0.$ Conversely, if $\norm{u}_{C^{k,\alpha}(\bar{\Omega})}=0$, then $\norm{u}_{C(\Omega)}=0$ with continuity of $u$, so $u=0$ on $\bar{\Omega}$.
\end{enumerate}
Therefore, $\norm{\cdot}_{C^{k,\alpha}(\bar{\Omega})}$ is a norm.


\begin{enumerate}
\item[(a)] Clearly, $0\in C^{k,p}(\bar{\Omega})$. For $f_1, f_2\in C^{k,p}(\bar{\Omega})$, $f_1+f_2\in C^k(\Omega)$ and $\norm{f_1+f_2}_{C^{k,\alpha}(\bar{\Omega})}\leq \norm{f_1}_{C^{k,\alpha}(\bar{\Omega})}+\norm{f_2}_{C^{k,\alpha}(\bar{\Omega})}< \infty$. Therefore, $f_1+f_2\in {C^{k,\alpha}(\bar{\Omega})}$. $f_1+f_2=f_2+f_1$ and for scalar $\lambda$, $\lambda f_1\in {C^{k,\alpha}(\bar{\Omega})}$ for $\norm{\lambda f_1}_{C^{k,\alpha}(\bar{\Omega})}=\abs{\lambda}\norm{f_1}_{C^{k,\alpha}(\bar{\Omega})}\leq \infty$. Therefore, $C^{k,p}(\bar{\Omega})$ is a vector space.
\item[(b)] Fix $x\in \Omega$ and take an open neighborhood $B(x,r)\subset \Omega$ for some $r>0$. Then there exists $N\in \mathbb{N}$ such that for $\frac{1}{N}<\epsilon$, then $B(x, \frac{1}{n})\subset \Omega$ for $n>N$. I'll use $C^\infty$ Urysohn lemma to show that there exists infinitely many linearly independent elements in $C^{k,\alpha}(\bar{\Omega})$. For $n>N$, take $K_n=\overline{B(x, \frac{1}{n+1})}$ and $U_n=B\left(x, \frac{1}{n+1}+\left(\frac{1}{n}-\frac{1}{n+1}\right)/2\right)$. Using $C^\infty$ Urysohn lemma, take $\phi^n\in C^\infty$ such that $1$ on $K_n$ and has support in $U$. Take finite elements in the set: $\{\phi^j\}_{j=N_1}^{N_n}$ with $N_i<N_j$ for $i<j$ and let $\sum\limits_{i=1}^n \lambda_i \phi^i=0$. For $x\in U_{N_1}\setminus B_{N_1+1}$, $\phi^{N_1}(x)=1$ but $\phi^{N_i}(x)=0$ for $i>1$. Therefore, $\lambda_1=0$. Repeating this argument, we can show that $\lambda_i=0$ for all $i$ and it means $\phi^n$ is linearly independent for all $n>N$ and consequently, $C^{k, \alpha}(\bar{\Omega})$ has infinite dimension.
\item[(c)] Let $\{u_i\}$ be a Cauchy sequence in $C^{k,p}(\bar{\Omega})$. For fixed $\epsilon>0$, there exists $N$ such that $i,j>N\Rightarrow \norm{u_i-u_j}_{C^{k,p}(\bar{\Omega})}\leq \epsilon$. It implies
\begin{equation*}
\begin{cases}
\norm{D^\alpha u_i-D^\alpha u_j}_{C(\bar{\Omega})}\leq \epsilon & \text{For }\abs{\alpha}\leq k \\
\left[D^\alpha u_i-D^\alpha u_j\right]_{C^{0,\gamma}(\bar{\Omega})}\leq \epsilon & \text{For }\abs{\alpha}=k.
\end{cases}
\end{equation*}
Since $D^\alpha u_i$ is uniformly Cauchy for $\abs{\alpha}\leq k$, $D^\alpha u_i$ converges to $u_\alpha$ for $\abs{\alpha}\leq k$ pointwisely. Also, these convergences are uniform. Therefore, $D^\alpha u=u_\alpha$. for all $\abs{\alpha}\leq k$.

Letting $i\rightarrow \infty$, $\left[D^\alpha u-D^\alpha u_j\right]_{C^{0,\gamma}(\bar{\Omega})}\leq \epsilon$ for $j>N$. Also,
\begin{equation*}
\begin{split}
\frac{\abs{D^\alpha u(x)-D^\alpha u(y)}}{\abs{x-y}^\gamma}-\sup_{\substack{x,y\in \Omega \\ x\neq y}}\left\{\frac{\abs{D^\alpha u_j(x)-D^\alpha u_j(y)}}{\abs{x-y}^\gamma}\right\}&\leq \frac{\abs{D^\alpha u(x)-D^\alpha u(y)}}{\abs{x-y}^\gamma}-\frac{\abs{D^\alpha u_j(x)-D^\alpha u_j(y)}}{\abs{x-y}^\gamma} \\
&\leq \frac{\abs{D^\alpha (u-u_j)(x)-D^\alpha (u-u_j)(y)}}{\abs{x-y}^\gamma} \\
&\leq \left[D^\alpha u-D^\alpha u_j\right]_{C^{0,\gamma}(\bar{\Omega})}\leq \epsilon
\end{split}
\end{equation*}
for all $x,y\in \Omega$, $x\neq y$. Therefore, $\frac{\abs{D^\alpha u(x)-D^\alpha u(y)}}{\abs{x-y}^\gamma}\leq [D^\alpha u_j]_{C^{0, \gamma}(\bar{\Omega})}+\epsilon$ and $[D^\alpha u]_{C^{0, \gamma}(\bar{\Omega})}<\infty$. Therefore, $u\in C^{k, \alpha}(\bar{\Omega})$. It means $C^{k, \alpha}(\bar{\Omega})$ is Banach space.
\end{enumerate}
\end{proof}
\section*{Problem 4}
\begin{proof}
Since $U$ is bounded, open subset of $\rr^n$, and $\partial \Omega$ is $C^1$,
\begin{equation*}
W^{1,p}(\Omega)\subset C^{0, \alpha}(\bar{\Omega}),~\norm{u}_{C^{0, \alpha}(\bar{\Omega})}\leq C\norm{u}_{W^{1,p}(\Omega)}
\end{equation*}
for $\alpha=1-n/p$ and $C$ depends only on $p,n$ and $\Omega$. Also, $C^{0, \alpha}(\bar{\Omega})\subset C^{0, \tilde{\alpha}}(\bar{\Omega})$ since $\norm{u}_{C(\bar{U})}$ is same for both norm and if $[u]_{C^{0, \alpha}(\bar{\Omega})}<\infty$, then as $\abs{x-y}\rightarrow 0$, $\frac{\abs{u(x)-u(y)}}{\abs{x-y}^{\tilde{\alpha}}}\rightarrow 0$ because $\frac{\abs{u(x)-u(y)}}{\abs{x-y}^{\alpha}}<\infty$ for all $x,y\in \Omega$, $x\neq y$, $[u]_{C^{0,\tilde{\alpha}}(\bar{\Omega})}<\infty$ and $u\in C^{0, \tilde{\alpha}}(\bar{\Omega})$.

Now, we need to show that each bounded sequence in $W^{1,p}(\Omega)$ is precompact in $C^{0, \alpha}(\bar{\Omega})$. Let a bounded sequence in $W^{1,p}(\Omega)$: $\{u_m\}_{m=1}^\infty$ and $\sup\limits_m\left\{\norm{u_m}_{W^{1,p}(\Omega)}\right\}=K$. 
By Morney's inequality, we can assume that $\{u_m\}\subset C^{0, \alpha}(\bar{\Omega})\subset C^{0, \tilde{\alpha}}(\bar{\Omega})$ and there exists constant $K'$ such that $\norm{u}_{C^{0, \alpha}(\bar{\Omega})}\leq K'$ for all $m$: For $\abs{x-y}\leq 1$, $\frac{\abs{u(x)-u(y)}}{\abs{x-y}^{\tilde{\alpha}}}\leq \frac{\abs{u(x)-u(y)}}{\abs{x-y}^{\alpha}}\abs{x-y}^{-\tilde{\alpha}+\alpha}\leq \frac{\abs{u(x)-u(y)}}{\abs{x-y}^{\alpha}} R^{-\tilde{\alpha}+\alpha}$ where $R$ is a constant such that $\Omega\subset B(0, R)$.

To use Arzela-Ascoli theorem, we need functions having compact domain. I'll denote $\bar{u_m}$ be a a function such that $\bar{u}_m=u_m$ in $\Omega$ and for $x\in \partial \Omega$, $u_m(x)=\lim\limits_{r\rightarrow 0} u_m(y)$ where $y\in B(x, r)\cap \Omega$. I'll show that $\bar{u}_m$ is continuous function on $\bar{\Omega}$. Fix $x\in \partial \Omega$. Since $u$ is bounded, $u_m(x)$ is uniformly bounded in $\partial \Omega$ if they exist. Fix $r>0$ and let $a=\lim\limits_{r\rightarrow 0}\left\{\inf\limits_{y\in B(x,r)\cap \Omega} u_m(y)\right\}$ and $b=\lim\limits_{r\rightarrow 0}\left\{\sup\limits_{y\in B(x,r)\cap \Omega} u_m(y)\right\}$. If $a\neq b$, then it means there exists $x,y\in \Omega$ such that $\abs{x-y}<r$ but $\abs{f(x)-f(y)}>(b-a)/2$ for all $r>0$ which is contradiction to continuity of $u$. Therefore, the limit $u_m(x)$ for $x\in \partial \Omega$ is well defined and $\bar{u}_m$ is continuous on $\bar{\Omega}$.

Let's check the condition for Arzela-Ascoli theorem for $\bar{u}_m$.
\begin{enumerate}
\item[1.] For each $m$, $\bar{u}_m$ is continuous on compact set $\bar{\Omega}$.
\item[2.] Since $\norm{\bar{u}_m}_{C(\bar{\Omega})}\leq K'$, $\{\bar{u}_m\}$ is pointwisely bounded.
\item[3.] Assume $\tilde{\alpha}>0$ $\frac{\abs{\bar{u}_m(x)-\bar{u}_m(y)}}{\abs{x-y}^{\alpha}}\leq K'$ for all $x,y\in \Omega$, $x\neq y$. Therefore, $\abs{\bar{u}_m(x)-\bar{u}_m(y)}\leq K'\abs{x-y}^\alpha$ for $x,y\in \bar{\Omega}$ for all $m$ and it means $\{u_m\}$ is equicontinuous on $\bar{\Omega}$.
\end{enumerate}
Therefore, we can use Arzela-Ascoli theorem and find a uniformly convergent subsequence $\{\bar{u}_{m_j}\}$ in $C^{0, \tilde{\alpha}}(\bar{\Omega})$ and it means $\{u_{m_j}\}$ is uniformly converges in $C^{0, \tilde{\alpha}}(\bar{\Omega})$. Since $C^{0, \tilde{\alpha}}(\bar{\Omega})$ is Banach space, the converging point is in $C^{0, \tilde{\alpha}}(\bar{\Omega})$. Hence,
\begin{equation*}
W^{1,p}(\Omega)\subset\subset C^{0, \tilde{\alpha}}(\bar{\Omega})
\end{equation*}
for all $\tilde{\alpha}\in (0, \alpha)$. If $\tilde{\alpha}=0$, then find $0<\tilde{\alpha}'<\alpha$ and do the same procedure above. Since $C^{0, \tilde{\alpha}'}(\bar{\Omega})\subset C^{0, 0}(\bar{\Omega})$, the above compact inclusion  is true for $\tilde{\alpha}=0$.
\end{proof}
\section*{Problem 5}
Fix $\epsilon>0$. Define $\Omega_\epsilon\coloneqq\{x\in \Omega|d(x,\partial \Omega)>\epsilon\}$. Let's mollify the $u$ with standard mollifier $\eta_\epsilon$ and denote it $u^\epsilon$. Then,
\begin{equation*}
Du^\epsilon=\eta_\epsilon*Du=0
\end{equation*}
in $\Omega_\epsilon$. It implies that if $B(x, r)\subset \Omega_\epsilon$ for small enough $r>0$, $u^\epsilon$ is constant on $B(x,r)$ since the derivative of $u^\epsilon$ is zero on the set. In other words, it is locally constant in $\Omega_\epsilon$.

Let $x\in U$ and $B(x, r)$ be an open neighborhood of $x$ in $\Omega$ and it is compactly embedded, then there exists $\epsilon$ such that $B(x, r)\subset \Omega_\epsilon$ and by previous, we know that $u^\epsilon$ is constant on $B(x,r)$. Let the constant value $c^\epsilon$. We know that $u^\epsilon\rightarrow u$ as $\epsilon\rightarrow 0$ and it means on $u$ is constant a.e. on $B(x,r)$.(If not, there always exists non measure zero set such that $u^\epsilon$ is different with $u$ on $B(x,r)$.) Also, it behave well since all any compactly embedded open neighborhood $B(x, r')$, the constant value should be same as $B(x,r)$ since $u$ is constant in $B(x,r')\cup B(x,r)$. Therefore, $u$ is locally constant function in a.e. sense).

Let take a partition such that $x\sim y$ if for $B(x,r_x)\subset\subset \Omega$ and $B(y, r_y)\subset\subset \Omega$, the constant values of the functions on the balls are same. Since $\Omega$ is locally constant, any element in partition is open set. Assume that there exists at least two element in the partition. This is impossible since $\Omega$ is connected set. Therefore, $u$ is a.e. constant function.
\section*{Problem 6}
First, I'll show that $u\in L^n(B_1(\bm{0}))$. Note that $u$ is symmetric function about rotation, so we can show that integral on $B_1(\bm{0})$ is finite by showing that integral is finite for $r$. Also, we can restrict the range of $r$ to $\left(0, \frac{1}{e-1}\right)$ since $u$ is bounded in outside of the range. In other words,
\begin{equation*}
\int_{B_1(\bm{0})}udx\leq C\int_0^{\frac{1}{e-1}} \abs{\log\log\left(1+\frac{1}{r}\right)}^{n}r^{n-1}dr
\end{equation*}
for some constant $C<\infty$. Let $y=\log\left(1+\frac{1}{r}\right)$, then
\begin{equation*}
\begin{split}
\abs{\int_0^{\frac{1}{e-1}} \left(\log\log\left(1+\frac{1}{r}\right)\right)^{n}r^{n-1}dr}&\leq\int_{1}^\infty \left(\log y\right)^n \frac{e^y}{(e^y-1)^{n+1}} dy \\
&\leq \int_{1}^\infty\left(\log y\right)^n \frac{2^{n+1}e^y}{e^{(n+1)y}}dy \\
&\leq \int_{1}^\infty y^n 2^{n+1}e^{-ny}dy<\infty
\end{split}
\end{equation*}
Therefore, $u\in L^n(B_1(\bm{0}))$, and $u\in L^1(B_1(\bm{0}))$.

Next, I'll show that $u$ has weak derivative in $B_1(\bm{0})$ and belongs to $L^n(B_1(\bm{0}))$. Since $u$ goes to $\infty $ as $x\rightarrow 0$, we need to care when we compute weak derivative. However, we can ignore at $\bm{0}$ by the following argument. Let $V$ be a compactly embedded set in $U$ and $\phi$ be a $C^\infty$ function having support $V$. Assume $\bm{0}\in V$. Without $\bm{0}$, $D u$ should be $\partial_{x_i}u$ for some $i$. Since $u$, $D^\alpha \phi$ for all $\alpha$ are $L^1$ function on $V$, we can use Fubini theorem, and rewrite the integral by
\begin{equation*}
\int_U uD \phi ~dx=\int_{-1}^1 (\cdots) dx_i
\end{equation*}
for $1\leq i \leq n$. Since $n>1$, we know that the $(n-1)$ dim plane through $0$ is measure zero set and it does not effect integral to delete $0$ from integral range of $x_1$. Therefore, the weak derivative is just derivative of $u$ except $\bm{0}$ More explicitly, for $\partial_{x_i}\phi$, take $j\neq i$. Then,
\begin{equation*}
\begin{split}
\int_U u\partial_{x_i}\phi dx&=\int_{(-1, 1)}\cdots \int_{x_1}^{x_2} u\partial_{x_i}\phi~ dx_i \cdots dx_j \\
&= \int_{(-1, 1)\setminus\{0\}}\cdots \int_{x_1}^{x_2} u\partial_{x_i}\phi ~dx_i \cdots dx_j\\
&= \int_{(-1, 1)\setminus\{0\}}\cdots \int_{x_1}^{x_2} \phi\partial_{x_i} u~ dx_i \cdots dx_j=\int_U \phi\partial_{x_i} u~dx.
\end{split}
\end{equation*}
Also, for any compact set not containing $\bm{0}$, we can just use $\int_U u\partial_{x_i}\phi~dx=\int_U \phi\partial_{x_i}u~dx$. Thus, $\partial_{x_i}u$ is weak derivative of $u$ except $\bm{0}$.

I'll show that $Du$ is in $L^n(B_1(\bm{0}))$. Computing partial derivative:
\begin{equation*}
\abs{\partial_{x_i} u}=\abs{\frac{1}{\log\left(1+\frac{1}{\abs{x}}\right)}\frac{1}{1+\frac{1}{\abs{x}}}\frac{x_i}{\abs{x}^3}}\leq  \frac{1}{\abs{\log\left(1+\frac{1}{r}\right)}}\frac{1}{r+1}\frac{1}{r}.
\end{equation*}
Then, by the same reason before, we just need to check whether the integral in finite for $r$ in $\left(0, \frac{1}{e-1}\right)$.
\begin{equation*}
\int_0^{\frac{1}{e-1}} \left(\frac{1}{\log\left(1+\frac{1}{r}\right)}\frac{1}{r+1}\frac{1}{r}\right)^n r^{n-1}dx \leq \int_0^{\frac{1}{e-1}} \left(\frac{1}{\log\left(1+\frac{1}{r}\right)}\right)^n \frac{1}{r}dr
\end{equation*}
Let $x=\log\left(1+\frac{1}{r}\right)$, then the integral becomes
\begin{equation*}
\int_1^\infty \frac{1}{x^n}\frac{e^x}{e^x-1}dx
\end{equation*}
For sufficiently large $R$, $\frac{e^x}{e^x-1}<2$ for $x>R$ and we know that $\int_1^\infty \frac{1}{x^n}$ converges for $n>1$. Therefore, $Du\in L^n(B_1(\bm{0}))$ and $u\in W^{1,n}(B_1(\bm{0}))$.
\section*{Problem 7}
Since $u\in L^2(\rr^n)$, $u=(\hat{u})^\vee$ by Theorem 2 in chapter 4.3 Evans. Then,
\begin{equation*}
\begin{split}
\abs{u(x)}&\leq \int_{\rr^n}\abs{e^{ikx}\hat{u}(k)}dk\leq \int_{\rr^n}\abs{\hat{u}(k)}dk \\
&=\int_{\rr^n}(1+\abs{k}^2)^{s/2}(1+\abs{k}^2)^{-s/2}\abs{\hat{u}(k)}dk \\
&\left(\leq \int_{\rr^n}(1+\abs{k}^2)^{s}\abs{\hat{u}}^2dk\right)^{1/2}\left(\int_{\rr^n}(1+\abs{k}^2)^{-s}dk\right)^{1/2}
\end{split}
\end{equation*}
For $\abs{k}>1$, $(1+\abs{k}^2)^s> \abs{k}^{2s}$ and
\begin{equation*}
\int_{\abs{k}>1} k^{-2s}dk=\sigma(S^{n-1})\int_1^\infty r^{-2s}r^{n-1}dr<\infty
\end{equation*}
since $-2s+n-1<-1$ and $\int_1^\infty r^\alpha dr<\infty$ for $\alpha<-1$. Therefore,
\begin{equation*}
\abs{u(x)}\leq C\left(\int_{\rr^n}(1+\abs{y}^2)^s \abs{\hat{u}}^2 dy\right)^{1/2}=C\norm{u}_{H^s(\rr^n)}
\end{equation*}
for some constant $C>0$ depends only on $s$ and $n$. This is true for a.e. $x$, so
\begin{equation*}
\norm{u}_{L^\infty(\rr^n)}\leq C\norm{u}_{H^s(\rr^n)}
\end{equation*}
\end{document}