\documentclass{article}
\usepackage{graphicx, amssymb}
\usepackage{amsmath}
\usepackage{amsfonts}
\usepackage{amsthm}
\usepackage{kotex}
\usepackage{bm}
\usepackage{hyperref}
\usepackage{xcolor}
\usepackage{mathrsfs}
\usepackage{mathtools}
\usepackage{physics}
\usepackage{ esint }

\textwidth 6.5 truein 
\oddsidemargin 0 truein 
\evensidemargin -0.50 truein 
\topmargin -.5 truein 
\textheight 8.5in

\DeclareMathOperator{\cc}{\mathbb{C}}
\DeclareMathOperator{\rr}{\mathbb{R}}
\DeclareMathOperator{\bA}{\mathbb{A}}
\DeclareMathOperator{\fra}{\mathfrak{a}}
\DeclareMathOperator{\frb}{\mathfrak{b}}
\DeclareMathOperator{\frm}{\mathfrak{m}}
\DeclareMathOperator{\frp}{\mathfrak{p}}
\DeclareMathOperator{\slin}{\mathfrak{sl}}
\DeclareMathOperator{\Lie}{\mathsf{Lie}}
\DeclareMathOperator{\Alg}{\mathsf{Alg}}
\DeclareMathOperator{\Spec}{\mathrm{Spec}}
\DeclareMathOperator{\End}{\mathrm{End}}
\DeclareMathOperator{\rad}{\mathrm{rad}}
\newcommand*\Laplace{\mathop{}\!\mathbin\bigtriangleup}
\newcommand{\id}{\mathrm{id}}
\newcommand{\Hom}{\mathrm{Hom}}
\newcommand{\Sch}{\mathbf{Sch}}
\newcommand{\Ring}{\mathbf{Ring}}
\newcommand{\T}{\mathcal{T}}
\newcommand{\B}{\mathcal{B}}
\newcommand{\Mod}[1]{\ (\mathrm{mod}\ #1)}
\newtheorem{lemma}{Lemma}
\newtheorem{theorem}{Theorem}
\newtheorem{proposition}{Proposition}

\begin{document}


\title{Partial Differential Equation - HW 3}
\author{SungBin Park, 20150462} 

 \maketitle

\section*{Problem 1}
I'll imitate the proof in Evans.
\begin{proof}
Since each $\Gamma_j$ is compact, $\partial \Omega$ is compact and we can choose finite points $x_i\in \partial \Omega$ with radius $r_i>0$ and $\partial \Omega\subset \cup_{i=1}^n B\left(x_i, \frac{r_i}{2}\right)$. If $x_i$ is not in end point of some $\Gamma_j$ for all $j$, then we can use the argument in the Evans, so we only need to consider the case that $x_i$ is in end point of $\Gamma_j$ for some $j$.

Fix $x^0$ is in end point of $\Gamma_j$ and assume that $x^0$ is also a end point of $\Gamma_{j+1}$. As $\Gamma_j$, $\Gamma_{j+1}$ are $C^1$, there exists $r_1,r_2>0$ and a $C^1$ function $\gamma_1$, $\gamma_2:\rr\rightarrow\rr$ implicit function theorem.

\section*{Problem 2}
\begin{enumerate}
\item[1.] $W_0^{1,p}(\Omega)$ is a vector space: For $f=0$, $f\in W_0^{1,p}(\Omega)$, so $W_0^{1,p}(\Omega)\neq \phi$. For $f_1, f_2\in W_0^{1,p}(\Omega)$, there exists $f_1^j, f_2^j$ such that $(f_1^j), (f_2^j)\in C_c^\infty (\Omega)$ and $f_1^j\rightarrow f_1$, $f_2^j\rightarrow f_2$ in $W^{1,p}(U)$. Since union of two compact set in $\Omega$ is compact in $\Omega$, $f_1^j+f_2^j\in C_c^\infty(\Omega)$ and for large enough $N$ satisfying $\norm{f_1^j-f_1}_{W^{1,p}(\Omega)} , \norm{f_2^j-f_2}_{W^{1,p}(\Omega)}\leq \epsilon/2$ for $j>N$, $\norm{f_1^j+f_2^j-f_1-f_2}_{W^{1,p}(\Omega)}\leq \norm{f_1^j-f_1}_{W^{1,p}(\Omega)}+\norm{f_2^j-f_2}_{W^{1,p}(\Omega)}\leq \epsilon$. Therefore, $f_1^j+f_2^j\rightarrow f_1+f_2$ and $f_1+f_2\in W^{1,p}(\Omega)$. Also, $\lambda f^j\rightarrow \lambda f$ in $W^{1,p}(\Omega)$ for scalar $\lambda$. Therefore, $W^{1,p}$ is vector space.(Other ...)
\item[2.] With the norm $\norm{\cdot}_{W^{1,p}(\Omega)}$, ${W_0^{1,p}(\Omega)}$ is Banach space: Let $f_j$ be a cauchy sequence in ${W_0^{1,p}(\Omega)}$. Since ${W^{1,p}(\Omega)}$ is Banach space, $f_j\rightarrow f$ in $W^{1,p}(\Omega)$. Since $\Omega$ is bounded and $\partial \Omega$ is $C^1$, there exists bounded linear operator $T:{W^{1,p}(\Omega)}\rightarrow L^p(\partial \Omega)$ and $Tf_j\equiv 0$ on $\partial U$ as $f_j\in {W_0^{1,p}(\Omega)}$. Then,
\begin{equation*}
\lim\limits_{j\rightarrow\infty}\norm{Tf_j-Tf}_{W^{1,p}(\Omega)}=\lim\limits_{j\rightarrow\infty}\norm{T(f_j-f)}_{W^{1,p}(\Omega)}\leq \lim\limits_{j\rightarrow\infty}\norm{T}_{W^{1,p}(\Omega)}\norm{f_j-f}_{W^{1,p}(\Omega)}=0
\end{equation*}
as $\norm{T}_{W^{1,p}(\Omega)}$ is bounded. Therefore, $Tf_j\rightarrow Tf$ and $\lim\limits_{j\rightarrow \infty}\norm{Tf_j}_{W^{1,p}(\Omega)}=\norm{Tf}_{W^{1,p}(\Omega)}=0$. As a result, $f\in W_0^{1,p}(\Omega)$ implying Cauchy sequence in ${W_0^{1,p}(\Omega)}$ converges.
\end{enumerate}
Therefore, $W_0^{1,p}(\Omega)$ is Banach space.
\section*{Problem 3}
For $k\in \mathbb{N}$ and $\alpha\in(0,1]$,
\begin{equation*}
C^{k,p}(\bar{\Omega})\coloneqq \{u\in C^{k}(\bar{\Omega}):\norm{u}_{C^{k,\alpha}(\bar{\Omega})}<\infty\}
\end{equation*}
\begin{enumerate}
\item[(a)] Clearly, $0\in C^{k,p}(\bar{\Omega})$. For $f_1, f_2\in C^{k,p}(\bar{\Omega})$,
\end{enumerate}
\section*{Problem 4}
\end{proof}
\end{document}